\chapter{参数估计与假设检验}

\section{参数估计的种类}
\label{sec:参数估计的种类}

\begin{itemize}
    \item 点估计
    \item 区间估计
\end{itemize}

\section{点估计}
\label{sec:点估计}

\subsection{矩估计}
\label{sub:矩估计}

总体 $X \sim f(x, \theta)$,但参数 $\theta$ 未知,需要对参数 $\theta$ 进行估计。
步骤
1. 取样$X_1, X_2, \cdots, X_n$
2. 计算样本均值 $\bar{X} = \frac{1}{n} \sum_{i=1}^n X_i$,根据大数定律有
\[
    \frac{1}{n} \sum_{i=1}^{n} X_i \to \frac{1}{n} \sum_{i=1}^n EX_i = EX
\]
3. 令 $\bar{X} = EX$,在 $EX$ 的结果中包含参数 $\theta \Rightarrow \hat{\theta}$

若含有两个参数 $\theta_1, \theta_2$ ,由大数定律知
\begin{enumerate}
    \item $\bar{X} \to EX, A_2 = \frac{1}{n} \sum X_i^2 \to \frac{1}{n} \sum EX_i^2 =  EX^2$
    \item 令$\bar{X} = EX, A_2 = EX^2$或令$\frac{1}{n} \sum \left(X_i-\bar{X}\right)^2=DX \Rightarrow \theta_1, \theta_2$ 的估计。
\end{enumerate}


\subsection{极大似然估计}
\label{sub:极大似然估计}

设 $X \sim f(x, \theta)$ 或 $X \sim f(x, \theta_1, \theta_2)$
步骤:
1.
\begin{enumerate}
    \item 对离散型
    \[
        L(x_1, x_2, \cdots, x_n; \theta) = \prod_{i=1}^n P\{X = x_i\}
    \]
    \item 对连续型
    \[
        L(x_1, x_2, \cdots , x_n; \theta) = \prod_{i=1}^n f(x_i; \theta)
    \]
\end{enumerate}
2. 求
\[
    \ln L(x_1, x_2, \cdots , x_n; \theta)
\]
3.
\begin{itemize}
    \item 对一个参数令
    \[
        \frac{\dif}{\dif \theta} \ln L = 0
    \]
    \item 对两个参数令
    \[
        \frac{\partial}{\partial\theta_i} \ln L = 0, \quad i = 1, 2
    \]
\end{itemize}
4. 解似然方程或方程组


\section{区间估计}
\label{sec:区间估计}

1. 设 $X \sim N(\mu, \sigma^2), \quad X_1, X_2, \cdots, X_n$ 是来自总体样本 $X$的一个样本。
若存在两个统计量 $\hat{\theta}_1\,,\hat{\theta}_2$,使得$P\{ \hat{\theta}_1 < \theta < \hat{\theta}_1\} = 1-\alpha$,称 $\left(\hat{\theta}_1,\hat{\theta}_2\right)$ 为参数 $\theta$ 的置信度为 $1-\alpha$ 的置信区间。
\begin{itemize}
    \item $\sigma^2$ 已知,$\mu$ 的置信度为 $1-\alpha$ 的置信区间为
    \[
        \left(\bar{X} - \frac{\sigma}{\sqrt{n}} u_{\frac{\alpha}{2}},
        \bar{X} + \frac{\sigma}{\sqrt{n}} u_{\frac{\alpha}{2}}\right)
    \]
    \emph{ 区间长度与样本无关 }
    \item $\sigma^2$未知,$\mu$的置信度为$1-\alpha$的置信区间为
    \[
        \left(\bar{X} - \frac{S}{\sqrt{n}} t_{\frac{\alpha}{2}}(n-1), \bar{X} + \frac{S}{\sqrt{n}} t_{\frac{\alpha}{2}}(n-1)\right)
    \]
    \emph{ 区间长度与样本有关 }
    \item $\mu$未知,$\sigma$的置信度为$1-\alpha$的置信区间为
    \[
        \left(\frac{(n-1)S^2}{\chi_{\frac{\alpha}{2}}^2(n-1)}, \frac{(n-1)S^2}{\chi_{\frac{1-\alpha}{2}}^2(n-1)}\right)
    \]
    \item $\mu$已知,$\sigma$的置信度为$1-\alpha$的置信区间
\end{itemize}

2. 三个评价标准

\begin{itemize}
    \item 参数估计的无偏性
    设 $\hat{\theta}$ 为 $\theta$ 的一个估计量,若 $E\hat{\theta} = \theta$,称 $\hat{\theta}$ 为 $\theta$ 的无偏估计量
    \item 参数估计的有效性
    设 $\hat{\theta}_1, \hat{\theta}_2$ 都是 $\theta$的无偏估计量, 若 $D(\hat{\theta}_1) < D(\hat{\theta}_2)$, 称 $\hat{\theta}_1$ 比 $\hat{\theta}_2$ 有效
    \item 一致性
    设 $\hat{\theta}$ 为 $\theta$ 的一个估计量,若 $\hat{\theta} \stackrel{P}{\longrightarrow} \theta$,称 $\hat{\theta}$ 为 $\theta$ 的一致估计
\end{itemize}

