\chapter{一维随机变量及其分布}

\section{随机变量、分布函数及性质}
\label{sec:随机变量_分布函数及性质}

\begin{enumerate}
    \item 随机变量$X(\omega)$,也就是概率$p = X(\omega)$
    \item 分布函数
\end{enumerate}

\section{离散型}
\label{sec:离散型}

$P\{X = x_i\} = p_i$

\begin{table}[htpb]
    \centering
    \caption{离散型随机变量$X$的分布律}
    \label{tab:离散型随机变量X的分布律}
    \begin{tabular}{|c|c|c|c|c|}
        \hline
        $X$ & $x_1$ & $x_2$ & \ldots & $x_n$ \\
        \hline
        $P$ & $p_1$ & $p_2$ & \ldots & $p_n$ \\
        \hline
    \end{tabular}
\end{table}

\[
    F(x) = P\{ X \leqslant x\} = \sum_{x_i \leqslant x} p_i
\]

\section{连续型}
\label{sec:连续型}

\[
    F(x) = \int_{-\infty}^x f(x) \dif x
\]

\section{常见的}
\label{sec:常见的}

\subsection{离散型}
\label{sub:离散型}

\subsubsection{二项分布}
\label{ssub:二项分布}

\begin{definition}[$X \sim B(n,p)$]
    \[
        P\{ X=k \} = C_n^k \,p^k\, (1-p)^{n-k}
    \]
\end{definition}

\subsubsection{Poisson分布}
\label{ssub:poisson分布}

\begin{definition}[$X \sim \pi(k)$]
    \[
        P\{X = k\} = \frac{\lambda^k}{k!} e^{-\lambda}
    \]
\end{definition}

\subsubsection{几何分布}
\label{ssub:几何分布}

\begin{definition}[$X \sim G(p)$]
    \[
        P\{X = k\} = (1-p)^{k-1} p
    \]
\end{definition}

\subsubsection{超几何分布}
\label{ssub:超几何分布}

\begin{definition}[$X \sim H(N, M, n$]
    \[
        P\{X = m\} = \frac{C_M^m C_{N-M}^{n-m}}{C_N^n}
    \]
\end{definition}

\subsection{连续型}
\label{sub:连续型}

\subsubsection{均匀分布$X \sim U(a,b)$}
\label{ssub:均匀分布}

\begin{tabular}{r|l}
    $ f(x) =
    \begin{dcases}
        \frac{1}{b-a} \\
        0, \text{other}
    \end{dcases} $ & $ F(x) =
    \begin{dcases}
        0 \\
        \frac{x-a}{b-a} \\
        1
    \end{dcases} $
\end{tabular}

\subsubsection{指数分布$X \sim E(\lambda)$}
\label{ssub:指数分布}

\begin{tabular}{l|r}
    $f(x) =
    \begin{dcases}
        \lambda \exp\left( -\lambda x \right) \\
        0
    \end{dcases} $ &
    $ F(x) =
    \begin{dcases}
        1 - \exp\left( -\lambda x \right) \\
        0
    \end{dcases} $
\end{tabular}

\subsubsection{正态分布$X \sim N(\mu, \sigma^2)$}
\label{ssub:正态分布}
\[
    f(x) = \frac{1}{\sqrt{2\pi\sigma}} \exp\left( - \frac{(x-\mu)^2}{2\sigma^2} \right)
\]

\begin{align}
    \varphi &= \frac{1}{\sqrt{2\pi}} \exp\left( - \frac{x^2}{2} \right) \\
    \varphi(x) &= \int_{-\infty}^x \varphi(t)\dif t
\end{align}

\section{函数分布}
\label{sec:函数分布}

\[
    Y = \varphi(X)
\]

\subsection{$X$为离散型}
\label{sub:_x_为离散型}

\[
    P\{Y = y_j\} = P\{\varphi(X) = y_j\} = \sum_{\varphi(x_i)=y_j} P\{X = x_i\}
\]

\subsection{$X$为连续型}
\label{sub:_x_为连续型}

\subsubsection{$Y$为离散型}
\label{ssub:_y_为离散型}

先求出 $Y$ 的可能取值,在通过 $X$ 的概率分布求出 $Y$ 的可能取值对应的概率。

\subsubsection{$Y$为连续型}
\label{ssub:_y_为连续型}

\[
    \begin{cases}
        F_Y(y) = P\{ Y \leqslant y \} =
        P \{ \varphi(X) \leqslant y \} =
        \int_{\varphi(x) \leqslant y} f_X(x) \dif x , \quad f_y(y) = F_Y^{\prime}(y) \\
        f_Y(y) = f_X \left[h(y)\right] | h^{\prime} |,
        \quad y = \varphi(x)  x = h(y)
    \end{cases}
\]
