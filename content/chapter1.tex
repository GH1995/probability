\chapter{随机事件与概率}

\section{基本概念}
\label{sec:基本概念}

\section{事件的运算与关系}
\label{sec:事件的运算与关系}

\section{概率的定义与基本性质}
\label{sec:概率的定义与基本性质}

\section{概率基本公式}
\label{sec:概率基本公式}

\subsection{条件概率}
\label{sub:条件概率}


在事件 $A$ 发生的情况下 $B$ 发生的概率
\[
    P(B|A) = \frac{P(AB)}{P(A)}
\]

\subsection{乘法公式}
\label{sub:乘法公式}

\[
    \begin{aligned}
        P(AB) &= P(A) P(B |  A) \\
        P(A_1 A_2 \cdots A_n) &= P(A_1)\, P(A_2 | A_1) \, P(A_3 | A_1 A_2) \cdots P(A_n | A_1 \cdots A_{n-1})
    \end{aligned}
\]

\section{事件的独立性}
\label{sec:事件的独立性}

\[
    P(AB) = P(A) P(B)
\]

\section{全概率公式与贝叶斯公式}
\label{sec:全概率公式与贝叶斯公式}

\subsubsection{完备事件组}
\label{ssub:完备事件组}

\[
    A_1 + A_2 + \cdots + A_n = \Omega
\]

\subsubsection{全概率公式}
\label{ssub:全概率公式}

\[
    P(B) = \sum P(A_i) P(B | A_i)
\]

\subsubsection{贝叶斯公式}
\label{ssub:贝叶斯公式}

\[
    P(A_k | B ) = \frac{P(A_k) P(B | A_k)}{\sum P(A_i) P(B | A_i)}
\]

\section{三种常见的概型}
\label{sec:三种常见的概型}

\subsection{古典概型}
\label{sub:古典概型}

\subsection{几何概型}
\label{sub:几何概型}

\subsection{$n$ 重贝努利试验}
\label{sub:n_重贝努利试验}

$B_k= \{ n \text{次试验中}A \text{出现}k\text{次} \}$
\[
    P(B_k) = C_n^k p^k (1-p)^{n-k}
\]
