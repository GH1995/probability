\chapter{大数定理与中心极限定理}

\section{切比雪夫不等式}
\label{sec:切比雪夫不等式}

\begin{gather}
    P\{\abs{X- \mu} \geqslant \epsilon\}  \leqslant \frac{DX}{\epsilon^2} \\
    P\{\abs{X- \mu} <  \epsilon\}  \geqslant 1- \frac{DX}{\epsilon^2}
\end{gather}

\section{大数定律}
\label{sec:大数定律}

\begin{itemize}
    \item 切比雪夫大数定律
    \item 独立同分布大数定律
    \item 辛钦大数定律
\end{itemize}

\begin{theorem}
    所有的大数定律都是指随机变量组在独立的条件下,若干个随笔建立的平均值依概率收敛到其数学期望的平均值,即
    \begin{align}
        \frac{1}{n} \sum_{i=1}^n X_i \xrightarrow{\text{独立}} E\left(\frac{1}{n}\sum_{i=1}^n X_i \right) = \mu \\
        \intertext{更进一步,}
        \frac{1}{n}\sum_{i=1}^n X_i^k \xrightarrow{\text{独立}} E\left(\frac{1}{n}\sum_{i=1}^n X_i^k \right) = \mu
    \end{align}
\end{theorem}

\section{中心极限定理}
\label{sec:中心极限定理}


